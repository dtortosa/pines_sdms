\documentclass[12pt,a4paper]{article} 
\usepackage[utf8]{inputenc}
\begin{document}


% latex table generated in R 3.4.3 by xtable 1.8-2 package
% Mon Jul 22 12:35:53 2019
\begin{tabular}{llc}
  \hline
Variables & Bioclim abbreviations & Sum of ranks \\ 
  \hline
Temperature Seasonality (standard deviation *100) & BIO4 & 593 \\ 
  Mean Temperature of Coldest Quarter & BIO11 & 740 \\ 
  Temperature Annual Range (BIO5-BIO6) & BIO7 & 745 \\ 
  Isothermality (BIO2/BIO7) (* 100) & BIO3 & 809 \\ 
  Min Temperature of Coldest Month & BIO6 & 893 \\ 
  Mean Temperature of Wettest Quarter & BIO8 & 1048 \\ 
  Annual Mean Temperature & BIO1 & 1191 \\ 
  Mean Temperature of Warmest Quarter & BIO10 & 1886 \\ 
  Mean Diurnal Range (Mean of monthly (max temp - min temp)) & BIO2 & 1904 \\ 
  Max Temperature of Warmest Month & BIO5 & 1904 \\ 
  Mean Temperature of Driest Quarter & BIO9 & 1935 \\ 
   \hline
Moisture of Driest Quarter & BIO17 & 1109 \\ 
  Moisture of Driest Month & BIO14 & 1161 \\ 
  Moisture of Warmest Quarter & BIO18 & 1180 \\ 
  Moisture of Coldest Quarter & BIO19 & 1384 \\ 
  Total (annual) moisture & BIO12 & 1471 \\ 
  Moisture Seasonality (standard deviation) & BIO15 & 1911 \\ 
  Moisture of Wettest Quarter & BIO16 & 1994 \\ 
  Moisture of Wettest Month & BIO13 & 2017 \\ 
   \hline
Clay content (\%) &  & 1418 \\ 
  Ph (index * 10) &  & 1649 \\ 
  Sand content (\%) &  & 1873 \\ 
  Silt content (\%) &  & 1933 \\ 
  Organic carbon (g/kg) &  & 2095 \\ 
  Absolute depth to bedrock (cm) &  & 2211 \\ 
  Cation-exchange capacity (CEC; cmolc/kg) &  & 2258 \\ 
   \hline
\end{tabular}

\end{document}